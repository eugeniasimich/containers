\chapter{Construcciones con containers}\label{chapter:construcciones}

\begin{epigraphs}
\qitem{
Se conservaron algunos adjetivos de hoy en día como bueno, fuerte, grande, negro, blando, pero en un número muy reducido. Por otra parte, su necesidad era mínima, ya que se llegaba a cualquier significado adjetival añadiendo {\it lleno} a un sustantivo-verbo. (...) Además, a cualquier palabra —y esto, como principio, se aplicaba a todas las palabras del idioma—, se le daba sentido de negación añadiendo el prefijo {\it in} o se le daba fuerza con el sufijo {\it plus}, o para aumentar el énfasis, {\it dobleplus}.}{1984\\ George Orwell}
\end{epigraphs}

En este capítulo presentaremos una serie de construcciones
que no solo aportarán a la modularidad de uso del universo de containers, apuntando a la programación genérica, sino que también clasificarán a la categoría de containers dentro de ciertas colecciones de categorías con propiedades interesantes, las categorías cartesianas cerradas.

Hemos visto en el capítulo \ref{cont} que los containers resultan ser una codificación válida de ciertos constructores de tipos. Desde este punto de vista propondremos formas genéricas de construir, por ejemplo, a partir de un par de constructores, el constructor de pares.
Pensando lo antedicho en términos algebraicos, podemos intuir que hablamos de un operador producto sobre containers y su clausura. Es decir, que los containers son cerrados bajo alguna definición de producto. Si bien en términos particulares eso es cierto, nos interesa asimismo tomar un punto de vista aún más general, retomando las nociones de teoría de categorías expuestas en las preliminares y continuadas en el capítulo \ref{cont}. De esta forma, veremos lo que significa que una categoría {\it cuente con productos}. Hecho esto, mostraremos que la categoría $\Cont$ de containers y sus morfismos, efectivamente cuenta con ellos.
Además del producto, presentaremos las construcciones de coproducto, exponencial, objetos inicial y terminal y veremos brevemente porqué se las denomina {\it construcciones universales}.

Debido a que no asumimos conocimientos previos de teoría de categorías, para cada construcción abstracta comenzaremos introduciendo brevemente su dimensión categórica. Para ello, se impulsará al lector a pensar desde lo particular --la teoría de conjuntos-- hacia lo general --la teoría de categorías-- y así motivar la introducción de cada nuevo concepto categórico a partir de la abstracción de un caso particular, siguiendo un camino similar al transitado por algunos libros de referencia~\cite{Awodey,BarrWells}.   
Con esta base expondremos una formalización genérica en código Agda, extendiendo lo presentado en la sección \ref{cat}. 
Finalmente, exhibiremos la forma que tienen dichas construcciones sobre containers. Cuando sea posible daremos también una interpretación menos abstracta, en teoría de la programación, siguiendo la línea de presentación del capítulo \ref{cont}, con ejemplos y análisis particulares al caso.

