\section{Coproducto}\label{cons:coprod}

\subsection{Motivación: (De)construyendo el coproducto}

Muchas de las construcciones de teoría de categorías pueden pensarse a partir de la teoría de conjuntos, con vistas a la generalización. En efecto, muchas de estas nociones categóricas han surgido de esta manera. 
Nuestra meta en esta sección de motivación es tratar de seguir ese proceso que va desde lo particular a lo general para así comprender más profundamente la construcción de {\it coproducto} que nos concierne analizar en esta parte del trabajo.

La unión disjunta de conjuntos, que notamos $\uplus$, es la modificación de la unión clásica de conjuntos donde se {\it etiquetan} los elementos según el conjunto desde el que provienen. Para mayor claridad, observemos el siguiente ejemplo, donde las {\it etiquetas} se muestran como subíndices:
\begin{center}
\begin{tabular}{c c c c}
  $A$ & $B$ & $A \cup B$ & $A \uplus B$ \\
  $\{ x , y \}$ & $\{ x, z, w \}$ &  $\{ x, y, z , w \}$ & $\{ x_{A} , y_{A}, x_{B}, z_{B}, w_{B} \}$
\end{tabular}
\end{center}


Notar que con esta modificación la operación deja de ser idempotente: unir disjuntamente un conjunto consigo mismo traerá como resultado la duplicación de sus elementos.

Un habitante de la unión disjunta de dos conjuntos será o bien un elemento del primer conjunto o bien uno del segundo. En el caso de los conjuntos, el coproducto nos permite codificar una opción, una disyunción.
En programación suele utilizarse esta construcción para expresar que un elemento es {\it o bien} de tipo $A$ {\it o bien} de tipo $B$.
 En Agda suele definirse como:

\begin{agdacode}{\it Unión disjunta de conjuntos}\label{code:uplus}
  
\ExecuteMetaData[latex/Aux.tex]{uplus}
\end{agdacode}

En resumidas cuentas, un elemento del conjunto $A$ \AgdaFunction{$\uplus$} $B$ es {\it o bien} un valor de la forma \AgdaInductiveConstructor{inj$_1$} $a$ (con $a\, :\, A$) {\it o bien} un valor de la forma \AgdaInductiveConstructor{inj$_2$} $b$ (con $b\, :\, B$). 
A las funciones de etiquetado suele designárseles los nombres de {\it inyecciones}, el nombre \AgdaInductiveConstructor{inj} proviene del vocablo inglés {\it injection}. Se las denomina de esa manera porque se considera que inyectan en el objeto coproducto la información de los sumandos.

Volviendo a teoría de categorías, hemos mencionado en la sección \ref{cat} que dicha teoría hace hincapié en las relaciones entre los objetos más que en los objetos en sí mismos. Dicho de otra forma, son los morfismos los que capturan la esencia de cada idea. Trataremos entonces de expresar cada concepto que nos interesa de la teoría de conjuntos centrándonos en las funciones que acompañan a cada construcción, es decir, en las relaciones entre los conjuntos involucrados.
Sin embargo, la definición recientemente expuesta hace justamente lo que queremos evitar: referirse a los habitantes de los conjuntos. Para comenzar a independizarnos de ellos podemos preguntarnos cómo definir un objeto coproducto a partir de las relaciones que podemos percibir entre los objetos.

En términos particulares, dados dos conjuntos $A$ y $B$, ¿qué relación tienen con su unión disjunta? ¿qué función involucra a $A$ y a $B$ con $A \uplus B$? 
Por empezar, siempre podremos construir funciones que vayan desde cada conjunto hacia su unión disjunta, las funciones de inyección.  
Con esto en mente, podríamos pensar entonces al coproducto como un objeto equipado con dos morfismos desde cada objeto factor hacia él.

Pero no cualquier objeto de estas características es el objeto coproducto, ya que claramente no cualquier conjunto $C$ y par de funciones que vayan de $A$ y $B$ hacia $C$ resulta ser la unión disjunta. Entonces, para indicar porqué otro objeto {\it no} es el coproducto, habrá que involucrar también a las relaciones entre el verdadero objeto coproducto que estamos definiendo y {\it cualquier otro} objeto de la categoría que se postule como tal. Siendo $C$ un conjunto cualquiera y $f: A \to C,\ g: B \to C $ funciones, ¿qué hace que $C$ no sea un buen candidato a ser el objeto coproducto? ¿qué relación encontramos entre $C$ y $A\uplus B$? Pues desde la unión disjunta {\it siempre} podremos construir una función hacia  este objeto $C$ simplemente eligiendo qué función aplicar, si $f$ o $g$. Aún más, dicha función es la única entre $A\uplus B$ y $C$ con estas características. La definimos como:


\begin{agdacode}\hspace{3ex}\label{code:choice}
  
\ExecuteMetaData[latex/Aux.tex]{choice}
\end{agdacode}

Luego de haber desmenuzado a la unión disjunta de conjuntos en aras de la generalidad categórica, continuaremos la exposición con la definición formal de coproductos en una categoría.

\subsection{Definición y formalización}
\begin{definition}\label{cat:coprod}
  Sean $A,B$ objetos de una categoría \C.
El {\it coproducto} de $A$ y $B$ es un objeto $A + B$, junto con dos morfismos
\flecha{A}{\iota_1}{A + B} y
\flecha{B}{\iota_2}{A + B} 
tal que para todo otro objeto $C$ y par de morfismos $\flecha{A}{f}{C},\ \flecha{B}{g}{C}$ de la categoría, exista un único morfismo $[ f,g ]$ que haga conmutar al siguiente diagrama
:
\begin{center}
  \xymatrixcolsep{3pc} \xymatrixrowsep{3pc}
  \centerline{\xymatrix{ 
      & C   & \\
      A \ar[ur]^{f} \ar[r]_{\iota_1} & A + B \ar@{-->}[u]_{[f,g]} & B \ar[l]^{\iota_2} \ar[ul]_{g}}}
\end{center}
\end{definition}

Los morfismos como $[f,g]$ suelen denominarse {\it mediales} por ser los que atraviesan el diagrama conmutativo. En el caso particular del coproducto, lo llamaremos morfismo de {\it coreunión}. La representación del morfismo como una flecha con línea de puntos en el diagrama conmutativo es un recurso para expresar unicidad. Es decir, para que el objeto $A + B$ cumpla con la definición, debe existir un único morfismo $\flecha{A+B}{}{C}$ que haga conmutar al diagrama.

Cuando para cada par de objetos de una categoría existe su objeto coproducto, decimos que la categoría {\it cuenta con coproductos}.

Habiendo definido al objeto coproducto podemos mencionar de qué se tratan las {\it construcciones universales}, al menos sin referirnos a los pormenores del asunto. En general, cuando hablamos del ``mejor objeto'' que cumpla con ciertas propiedades, como lo hicimos con el coproducto, nos estamos refiriendo a una construcción universal. Muchas veces con ``mejor'' nos referimos a menor, mayor, con menos restricciones, etc.

Las propiedades universales son ubicuas en la matemática. La teoría de categorías apunta a analizarlas desde un punto de vista abstracto y así comprenderlas en su totalidad, evitando repetir el estudio de cada instancia en particular. Por ejemplo, sabemos que las propiedades universales definen objetos que resultan ser únicos, salvo isomorfismos únicos. Particularmente, puede demostrarse que todo objeto coproducto de otros dos es único hasta isomorfismos. Esto significa que es posible que exista más de un objeto coproducto dados otros dos, pero todos ellos serán isomorfos. 

\vspace{3ex}

Como nota de color resulta interesante mencionar que otra denominación para los coproductos que suele encontrarse en la bibliografía es la de {\it suma}, siendo un nombre más intuitivo, sobre todo cuando pensamos en ciertas categorías. El apelativo {\it coproducto} se origina en el hábito de denominar co-{\it concepto} al dual de un {\it concepto}. Recordemos que la categoría dual $\C^{op}$ de una categoría $\C$ es aquella que se obtiene al revertir el sentido de los morfismos. De este modo, un co-{\it concepto} en una categoría $\C$ es un {\it concepto} en $\C^{op}$. Veremos en la sección \ref{cons:prod} que la idea de {\it producto} puede pensarse en los mismos términos que el coproducto, pero revirtiendo el sentido de las flechas del diagrama conmutativo. Dicho esto cabe preguntarse: ¿porqué entonces no se llama {\it cosuma} al producto? Pues porque en teoría de categorías, el producto fue concebido primero. En 1950, Saunders Mac Lane~\cite{maclane1950} presenta lo que sería muy probablemente el ejemplo más remoto de uso de la teoría de categorías para definir una noción matemática fundamental\footnote{Ver~\cite{Awodey}}. 

\vspace{3ex}

A continuación formalizaremos en Agda lo presentado hasta aquí. Seguiremos los pasos de la sección \ref{cat} y extenderemos dicho modelo para permitir expresar cuándo una categoría tiene coproductos.

Como expusimos anteriormente, Agda nos permite definir estructuras algebraicas que incluyan no sólo al portador con sus operaciones sino también las propiedades que se deben cumplir.
El código \ref{code:hasCoproducts} formaliza lo presentado, definiendo una estructura de record indexado por una categoría \AgdaBound{\C}. Dar una instancia de \AgdaFunction{HasCoproducts} \AgdaBound{\C} implica probar que la categoría \AgdaBound{\C} cuenta con coproductos.

\begin{agdacode}{\it Formalización de categoría con coproductos}\hspace{3ex}\label{code:hasCoproducts}
  
\ExecuteMetaData[latex/Cat.tex]{hasCoproducts}
\end{agdacode}

El elemento \AgdaField{Coprod} es el que construye cada objeto coproducto, dados otros dos. Los elementos \AgdaField{inl} e \AgdaField{inr} son morfismos en la categoría \AgdaBound{\C}, definidos para todo par de objetos \AgdaBound{X, Y} como las respectivas inyecciones. De forma similar se expresa la signatura del morfismo medial \AgdaField{copair}. 
Los campos dedicados a las pruebas garantizarán que las construcciones proporcionadas son las correctas, dado que hacen conmutar al diagrama de la definición \ref{cat:coprod}.



 Veamos a continuación cómo se manifiesta esta construcción en la categoría de containers.

\subsection{Coproducto de Containers}

Para formalizar el coproducto en la categoría de containers necesitamos contar con un elemento \AgdaField{Coprod} que arme un container a partir de otros dos. Será obligatorio definir también funciones \AgdaField{inl}, \AgdaField{inr} y \AgdaField{copair} que proyecten cada uno de los containers involucrados en un coproducto. Finalmente, habrá que suministrar las pruebas requeridas y así garantizar que estamos presentando la construcción correcta.  
En resumen, buscamos construir un elemento \AgdaFunction{ContHasCoproducts}, llenando los huecos del código \ref{cont:coprod:huecos} indicados con signos de interrogación y así formalizar la noción de que la categoría \AgdaFunction{$\Cont$} presentada en la sección \ref{cont:cat} tiene coproductos.
\begin{agdacode}\hspace{1ex}\label{cont:coprod:huecos}
  
\ExecuteMetaData[tex/AuxCatCont.tex]{hasCoproducts}
\end{agdacode}

A continuación se muestra el código del constructor \AgdaFunction{Either} que arma un coproducto a partir de dos containers $C$ y $D$. Por un lado, las formas de un container coproducto serán o bien formas de $C$ o bien formas de $D$. Por otro lado, hecha esa elección y determinada una forma $s$, las posiciones posibles dentro de ella dependerán del conjunto del que $s$ proviene. Si $s$ es una forma de $C$, entonces las posiciones posibles serán las posiciones de $C$ en $s$ y de forma similar en el caso en que $s$ sea forma de $D$.

\ExecuteMetaData[latex/Sum.tex]{either}

Renombramos como \AgdaFunction{$[\_,\_]_{\Set}$} al operador \AgdaFunction{[\_,\_]} presentado en el código \ref{code:choice} para dejar en claro que se trata del correspondiente a la categoría $\Set$ y no al que presentaremos próximamente, perteneciente a la categoría $\Cont$.

Lógicamente, las inyecciones serán morfismos en la categoría de containers. La primera componente del morfismo es la función sobre formas, que simplemente inyecta una forma de $C$ en el conjunto \Sh $C$ $\uplus$ \Sh $D$. Dada una forma $c$ de $C$, las posiciones en \AgdaFunction{Either} $C$ $D$ se reducen a posiciones de $C$ en $c$. La función de formas será simplemente la función identidad.

\ExecuteMetaData[latex/Sum.tex]{inj1}

La segunda inyección es análoga.

El morfismo de coreunión construye un morfismo a partir de otros dos, eligiendo aplicar uno u otro dependiendo del caso. La función de formas elegirá cuál de las respectivas funciones de formas aplicar. La nueva función de posiciones hará lo mismo con las respectivas funciones de posiciones originales. 

\ExecuteMetaData[latex/Sum.tex]{coreunion}

Para terminar, se requiere dar pruebas de haber construido el coproducto correctamente. Probaremos por un lado que
inyectar elementos al lado izquierdo de un coproducto para luego aplicar la coreunión es igual a sólo hacer lo que decide la coreunión sobre la parte izquierda. Igualmente sobre la parte derecha.
Es decir, queremos construir elementos de \AgdaFunction{$[$} $f , g$ \AgdaFunction{$]\ \circ\ \iota_{1} \cong$} $f$ y de \AgdaFunction{$[$} $f , g$ \AgdaFunction{$]\ \circ\ \iota_{2} \cong$} $g$ dados los morfismos $f$ y $g$. A continuación se expone la definición para el primer caso, la prueba para la segunda proyección es análoga.

\ExecuteMetaData[latex/Sum.tex]{copairIdl}



Recordemos que el único habitante expreso de la equivalencia proposicional es \AgdaInductiveConstructor{refl}. En el caso de esta primera prueba, Agda computa cada lado de la equivalencia y obtiene el mismo resultado, pudiendo entonces probar la regla trivialmente.

Pero hay casos donde de cada lado de la igualdad hay elementos de distinto tipo. Habrá que proveer primero una prueba de la equivalencia de esos tipos para poder continuar. Este es un caso muy común cuando tratamos con tipos indexados, como lo son el tipo de los containers y de los morfismos de containers. Para sortear este obstáculo se definen lemas que ayuden con la prueba, como ser el lema de equivalencia de morfismos expuesto en el código \ref{morph:equivalence}. Otro recurso al que habrá que apelar es al axioma de extensionalidad definido en \ref{code:ext}, en la situación en que queramos probar la equivalencia de funciones.

Apelando a los recursos recientemente mencionados, podremos demostrar que la coreunión es única. Homólogamente, probaremos que cualquier otro morfismo $h$ entre los elementos indicados que haga conmutar al diagrama de la definición \ref{cat:coprod} es equivalente a él. Es decir, dadas pruebas de que $h$ {\it se comporta} como una función de coreunión, veremos que en efecto lo es. Construiremos un elemento de $h$ \AgdaFunction{$\cong\ [$} $f\, ,\, g$ \AgdaFunction{$]$} a partir de elementos de
$h$ \AgdaFunction{$\circ \iota_1$}  \AgdaFunction{$\cong$} $f$ y de $h$ \AgdaFunction{$\circ\ \iota_2$}  \AgdaFunction{$\cong$} $g$. 

\ExecuteMetaData[latex/Sum.tex]{copairUnique}

Finalmente, habiendo presentado cada construcción y probado cada lema, podemos completar los huecos del código \ref{cont:coprod:huecos}, simbolizados con signos interrogatorios:

\begin{agdacode}{\it Formalización $\Cont$ como categoría con coproductos} 

\ExecuteMetaData[latex/CatCont.tex]{hasCoproducts}

\end{agdacode}

\subsubsection{Ejemplo con coproductos de containers}

Como ya hemos mencionado, el tipo de datos de coproducto se utiliza en programación para expresar que un elemento puede ser, o bien de un tipo, o bien de otro. En particular, suele utilizarse en los casos donde una función retorna o bien un valor, el valor esperado, o bien un mensaje de error en el caso en que un error haya ocurrido. Desde esta perspectiva, el constructor \AgdaFunction{Either} resulta una buena generalización de \AgdaDatatype{Maybe}, donde en lugar de retornar un valor único \AgdaInductiveConstructor{nothing} en el caso de error, se retorna una descripción más útil de lo ocurrido. 

En el código a continuación vemos un ejemplo sencillo de cómo con containers podemos crear un mecanismo para equipar a un elemento de tipo \AgdaDatatype{Maybe} $C$ en un valor de la forma \AgdaDatatype{Either} $C$ {\it Str}.
  
\ExecuteMetaData[latex/Examples.tex]{errorMes}

La función \AgdaFunction{errorMes} toma como argumentos un container $C$, un tipo {\it Str} y un valor de dicho tipo que será el elemento que se retorne en caso de error. El tipo {\it Str} también será determinado por el usuario. A pesar de que el nombre elegido evoca al tipo {\it String} de cadenas de caracteres, podría ser cualquier tipo que se considere apto para representar el error. Con esa información se construye un morfismo de containers que tiene por origen al container \AgdaDatatype{Compose} \AgdaFunction{cMaybe} $C$.

Recordemos que el container \AgdaFunction{cMaybe} tenía por conjunto de formas a los booleanos, siendo \AgdaInductiveConstructor{false} el representante del elemento canónico \AgdaInductiveConstructor{nothing} y el valor \AgdaInductiveConstructor{true} el representante de constructor \AgdaInductiveConstructor{just}.
De esta forma, el morfismo transforma el elemento \AgdaInductiveConstructor{false} en el nuevo valor $s$, dejándolo del lado derecho de la unión disjunta con el constructor \AgdaInductiveConstructor{inj$_2$}; del elemento \AgdaInductiveConstructor{true} se extrae la información de tipo $C$ almacenada y se la ubica del lado izquierdo de la unión disjunta. 
