\chapter{Hoja de ruta del código fuente}\label{guia}

La contribución principal del presente trabajo es una librería de containers extendida con las formalizaciones de la clausura de la categoría con respecto al producto, coproducto, exponencial y la existencia de objetos terminal e inicial. Actualmente está alojada en un repositorio {\it Git} situado en el siguiente enlace: \gitcode. Allí se puede encontrar todo el código Agda analizado en este trabajo, así como también porciones de código menos relevantes que no fueron expuestas, e incluso este documento. A continuación se expondrá un pantallazo general del contenido del mencionado repositorio. Para visualizar más claramente su estructura, observar la figura \ref{fig:map} (pág. \pageref{fig:map}).

En la carpeta \texttt{doc} se puede encontrar el texto completo de esta tesina, cuyo archivo \LaTeX\ principal es \texttt{main.tex}, compilable realizando \texttt{make}. En la subcarpeta \texttt{code} se encuentra el código simplificado para su exposición y anotado para la compilación del documento.

El resto de los archivos corresponden a la formalización en sí. Los archivos relacionados a la formalización de la teoría de categorías se encuentran en el archivo \texttt{Category.agda} y en la carpeta \texttt{Category}. El archivo \texttt{Extras.agda} contiene definiciones auxiliares utilizadas en muchas de las demostraciones. La formalización de containers se encuentra distribuida en los archivos almacenados en la carpeta \texttt{Container} y los archivos \texttt{Container.agda}.  
Cada una de las construcciones sobre containers se encuentra en un archivo distinto.

\begin{figure}[h]
\dirtree{%
.1 /.
  .2 \textcolor{blue}{doc}.
    .3 main.tex.
    .3 \textcolor{blue}{code}.
  .2 Extras.agda.
  .2 Category.agda.
  .2 \textcolor{blue}{Category}.
    .3 Coproduct.agda.
    .3 \textcolor{blue}{Examples}.
      .4 Fun.agda.
      .4 Sets.agda.
    .3 Exponential.agda.
    .3 Funtor.agda.
    .3 HomFunctors.agda.
    .3 Initial.agda.
    .3 Isomorphism.agda.
    .3 Natural.agda.
    .3 Product.agda.
    .3 Representables.agda.
    .3 Terminal.agda.
  .2 Container.agda.
  .2 \textcolor{blue}{Container}.
    .3 Composition.agda.
    .3 Coproduct.agda.
    .3 Examples.agda.
    .3 Exponential.agda.
    .3 Initial.agda.
    .3 Product.agda.
    .3 Terminal.agda.
}
\caption{Organización de los directorios del proyecto}\label{fig:map}
\end{figure}
