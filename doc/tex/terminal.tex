\section{Inicial y terminal}
\subsection{Motivación: (De)construyendo los objetos inicial y terminal}
\label{mot:initterm}
En las diversas secciones de motivación de cada construcción universal teníamos como objetivo el introducir la noción categórica a partir de pensar en teoría de conjuntos, yendo de lo particular a lo general. 

¿Cómo hablar del conjunto vacío sin referirnos a sus elementos, o mejor dicho, a la ausencia de ellos? La estrategia es siempre pensar en los morfismos, o en este caso, en las funciones; podemos entonces reformular la pregunta: ¿Qué relaciones se pueden establecer entre el conjunto vacío y el resto de los conjuntos?, ¿qué funciones podemos construir desde o hacia el conjunto vacío?

Hacia el conjunto vacío resulta imposible construir cualquier función.
Si pensamos por un momento e intentamos construir dicha función, nos resultará imposible, puesto que el conjunto vacío carece de elementos que oficien de recorrido de la función.

Por otro lado, es muy simple construir una función {\it desde} el conjunto vacío hacia cualquier otro objeto, simplemente por el hecho de no necesitar proveer ningún valor. La función vacía será la única función que podamos construir. Hemos encontrado una característica del conjunto vacío: siempre existe una única función desde él hacia cualquier otro conjunto. Es este atributo el que le da el nombre de {\it objeto inicial}
a su generalización en teoría de categorías. En la definición \ref{cat:initial} se expone formalmente.

Como ya hemos comentado en reiteradas ocasiones, si revertimos el sentido de los morfismos en una categoría $\C$ obtenemos una nueva categoría, la categoría dual $\C^{op}$. Allí encontramos los duales de cada construcción, en particular, de la construcción de objeto inicial: el objeto {\it terminal}. Si dijimos que un objeto es inicial cuando siempre exista un morfismo único hacia cualquier otro objeto, uno terminal será aquel con morfismos únicos {\it desde} cualquier otro objeto. En la categoría $\Set$ contamos con muchos objetos terminales. Cualquier conjunto con un único elemento posee la propiedad de aceptar una única función desde cualquier otro conjunto hacia él: la función constante.


\subsection{Definición y formalización}
\begin{definition}\label{cat:initial}

 Un objeto $\bot$ en una categoría $\C$ se dice {\it inicial} si por cada objeto $A$ de $\C$ hay exactamente un morfismo $\flecha{\bot}{}{A}$ desde él, simbolizado ``!`''.
\begin{center}
  \xymatrixcolsep{3pc}
  \centerline{\xymatrix{ 
      \bot \ar@{-->}[r]^{\rotatebox[origin=c]{-180}{$^{!}$}}  & A  }}
\end{center}

\end{definition}

\begin{definition}\label{cat:term}

 Un objeto $\top$ en una categoría $\C$ se dice {\it terminal} si por cada objeto $A$ de $\C$ hay exactamente un morfismo $\flecha{A}{}{\top}$ hacia él, simbolizado ``!''.
\begin{center}
  \xymatrixcolsep{3pc}
  \centerline{\xymatrix{ 
      A \ar@{-->}[r]^{!} & \top}}
\end{center}

\end{definition}


Damos a continuación la implementación en Agda de estos conceptos. Una categoría $\C$ : \AgdaDatatype{Category} tendrá objeto inicial si podemos proveer una instancia del record definido en el código \ref{code:cat:initial}.  Para ello habrá que individualizar un objeto \AgdaField{Initial}. Además, para poder proclamar la existencia de un morfismo único desde el objeto inicial hacia todo otro objeto, se requiere una forma de construir un habitante de \AgdaField{Hom} $\C$ \AgdaField{Initial} $X$, para todo $X$ y una prueba de su unicidad.

\begin{agdacode}{\it Formalización de categoría con objeto inicial}\hspace{3ex}\label{code:cat:initial}
  
\ExecuteMetaData[latex/Cat.tex]{hasInitial}
\end{agdacode}

Asimismo, proveer una instancia de \AgdaRecord{HasTerminal} $\C$, record definido en el código \ref{code:cat:terminal}, es prueba formal de la existencia de objeto terminal en la categoría $\C$.

\begin{agdacode}{\it Formalización de categoría con objeto terminal}\hspace{3ex}\label{code:cat:terminal}
  
\ExecuteMetaData[latex/Cat.tex]{hasTerminal}
\end{agdacode}

Para ver formalizado lo analizado al comenzar la sección, se expone a continuación los objetos inicial y terminal de la categoría de conjuntos.


\begin{agdacode}{\it Formalización de $\Set$ como categoría con objeto inicial}\hspace{3ex}\label{code:set:initial}
  
\ExecuteMetaData[latex/CatSet.tex]{setHasInitial}
\end{agdacode}

\begin{agdacode}{\it Formalización de $\Set$ como categoría con objeto terminal}\hspace{3ex}\label{code:set:terminal}
  
\ExecuteMetaData[latex/CatSet.tex]{setHasTerminal}
\end{agdacode}

Observemos que para el caso del objeto terminal, expuesto en el código \ref{code:set:terminal}, podríamos haber elegido cualquier conjunto de un elemento. La función \AgdaField{toTermMor} será simplemente la función que retorne el único elemento del conjunto. La garantía de que dicha función es única la provee el campo \AgdaField{isUniqueToTermMor}.


\subsection{Containers vacío y unitario}


%%%%%INICIAL

En esta sección proveeremos instancias de las formulaciones recientemente expuestas, para el caso de la categoría $\Cont$ de containers.

Si pensamos en que tenemos como objetivo proveer un container inicial tal que siempre exista un morfismo único desde él hacia cualquier otro objeto, entonces podemos empezar pensando en los morfismos. Un morfismo de containers se compone, por un lado, de una función entre formas y por otro de una función en posiciones. Quisiéramos que la construcción a proveer posea inherentemente una única función en formas y, elegida una forma, una única en posiciones. De lo analizado en la sección \ref{mot:initterm} de motivación, podemos intuir que el conjunto de las formas deberá ser el vacío. Llamamos \AgdaFunction{Zero} al container inicial:

\ExecuteMetaData[latex/Empty.tex]{Zero}

Como hemos expresado, el morfismo único constará de dos funciones únicas, las funciones vacías. Lo llamamos \AgdaFunction{Zero$_{m}$}.

\ExecuteMetaData[latex/Empty.tex]{Zerom}

Proveemos también la demostración de la unicidad del morfismo \AgdaFunction{Zero$_{m}$}, apelando al lema \AgdaFunction{mEq} y a los postulados de extensionalidad, utilizados de forma trivial con las funciones vacías. 

\ExecuteMetaData[latex/Empty.tex]{ZeromUnique}


En el código \ref{hasInitial} se presenta la instancia completa, garantizando que la categoría $\Cont$ cuenta con objeto inicial. 

\begin{agdacode}{\it Formalización de $\Cont$ como categoría con objeto inicial} \label{hasInitial}

  \ExecuteMetaData[latex/CatCont.tex]{hasInitial}
\end{agdacode}


Notar que así como hemos pensado en construir el objeto inicial de la categoría de containers a partir de la definición e intuición expresadas previamente, podríamos también haberlo motivado como la introducción de aquel container que representa al tipo vacío. Pensado de esa forma, observemos que ya contamos con una forma de construirlo, puesto que el container vacío no es más que el container constante presentado en el ejemplo \ref{cont:k}, instanciado en el conjunto vacío:

\sangrar
\AgdaFunction{Zero} $=$ \AgdaFunction{cK} $\bot$

%%%%%TERMINAL

De la misma forma, el container unitario es aquel que construye el tipo que alberga a un único elemento. Podemos pensarlo como un container constante instanciado en el tipo $\top$ con un único elemento, \AgdaInductiveConstructor{tt}.

\sangrar
\AgdaFunction{One} $=$ \AgdaFunction{cK} $\top$

Sin embargo, es posible demostrar que cualquier conjunto con un único elemento funciona igual de bien, con lo que podemos concluir que el objeto terminal no es único en la categoría de containers, como no lo era en $\Set$.
En teoría de categorías, las restricciones sobre los objetos siempre se dan a través de restricciones sobre los morfismos.
Esto es así porque el nivel de abstracción no nos permite analizar cada objeto en particular más que a partir de observar su relación con otros objetos. Esta es la razón por la que no hablaremos de igualdad entre los objetos. Serán los isomorfismos, i.e. ciertos morfismos particulares, los que otorgarán una forma de referirnos a objetos que comparten propiedades equivalentes.   
Es posible demostrar que el objeto terminal, tanto como el inicial, siempre que existan serán únicos salvo isomorfismos únicos.

Retomando la exposición en la categoría de containers, si desplegamos la definición de \AgdaFunction{cK}, vemos que \AgdaFunction{One} es el container con forma única, sin posiciones.

\ExecuteMetaData[latex/Unit.tex]{One}


El morfismo único hacia el container terminal resulta del morfismo único hacia el conjunto unitario por el lado de las formas y la función vacía por el lado de las posiciones.

\ExecuteMetaData[latex/Unit.tex]{Onem}

A continuación, la prueba de la unicidad de esta construcción. 

\ExecuteMetaData[latex/Unit.tex]{OnemUnique}


Finalmente, en el código \ref{hasTerminal} proveemos una instancia de \AgdaFunction{HasTerminal} $\Cont$:

\begin{agdacode}{\it Formalización de $\Cont$ como categoría con objeto inicial} \label{hasTerminal}

\ExecuteMetaData[latex/CatCont.tex]{hasTerminal}
\end{agdacode}
